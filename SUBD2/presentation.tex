\documentclass{beamer}
\usepackage[utf8]{inputenc}
\usepackage[T1]{fontenc}
\usepackage[bulgarian]{babel}
\usepackage{alltt}

\useoutertheme{shadow}

\setbeamercolor{title}{fg=red!80!black}
\setbeamercolor{frametitle}{fg=red!80!black}

\usetheme[secheader]{Madrid}
\usecolortheme{crane}

%Тема 11.
%7.2 Constraints on Attributes and Tuples
%• Not-Null Constraints
%• Attribute-Based CHECK Constraints
%• Tuple-Based CHECK Constraints
%• Modification of Constraints
%• Giving Names to Constraints
%• Altering Constraints on Tables

\title[Mногомерни данни в MySQL]{Многомерни и битови индекси. Дървовидни структури за многомерни данни в MySQL}
\author{Валентина Динкова, ф.н.71112}
\institute{ФМИ}
\date{\today}
\begin{document}
\begin{frame}
  \titlepage
\end{frame}
%\begin{frame}
  %\frametitle{Съдържание}
  %\tableofcontents
%\end{frame}

\begin{frame}
  \frametitle{GIS и разширението на MySQL за пространствени данни}
\begin{itemize}
 \item Какво е \textbf{GIS} и какво е \textbf{OGC}?
\end{itemize}
\textbf{GIS} означава Географска Информационна Система и е един от най-очевидните примери за пространствени данни.
\newline
\newline
\textbf{OGC} (Open Geospatial Consorcium) е организация, която работи по стандартизирането на различни области на GIS.
Един такъв стандарт е и спецификацията за SQL, която определя разширението на SQL базирани релационни бази данни,
 което да използва GIS обекти и операции.
\end{frame}

\begin{frame}
 OGC работи в 4 важни области:
\begin{itemize}
 \item типове данни;
 \item операции;
 \item възможност да се подават като вход и да се извеждат GIS данни;
 \item индексиране на пространствени данни.
\end{itemize}
Друга важна област са метаданните.
\end{frame}

\begin{frame}
\frametitle{Стандартът, използван от почти всички SQL бази данни с пространствено разширение, включително и MySQL}
\begin{center}
\includegraphics[width=60mm]{gis-datatypes.png}\end{center}
Типовете, отбелязани в сиво са абстрактни и обекти от тези типове не могат да се създават.
\end{frame}

\begin{frame}
 \frametitle{Пространствени индекси}
\begin{itemize}
\item Пространствените данни могат да се индексират също както останалите данни в MySQL. Но за да бъде 
индексирането ефективно, се използва пространствен тип индексиране, реализирано чрез R-дървета. 
MySQL използва \alert{R-дървета с квадратично разделяне};
\item Не всички engine-и поддържат многомерни индекси.
\end{itemize}
\end{frame}

\begin{frame}
\frametitle{R-дървета с квадратично разделяне}
Добавяме нов запис.
Нека сме намерили листото, където трябва да добавим новия запис и 
нека $M = $ \textit{``брой региони в листо''}
\begin{itemize}
\item Избираме 2 от $M+1$ записа да бъдат първите елементи на двете нови листа, като избираме двойката,която би заела
най-много място ако и двата елемента се постават на едно място (двойката при която покриващия регион ще е най-голям).
Намираме тази двойка като от областта покриваща двата записа изваждаме самите записи и искаме тази разлика да е най-голяма.
\begin{center}
\includegraphics[width=70mm]{Diagram1.png}\end{center}
\end{itemize}
\end{frame}

\begin{frame}
\begin{itemize}
\item Останалите записи разделяме в двете листа един по един.
На всяка стъпка разширяването, необходимо за добавянето на всеки от оставащите записи към всяко листо се изчислява
и добавеният запис е този, който е показал най-голяма разлика спрямо двете листа.
\end{itemize}
\end{frame}

%\begin{beamerboxesrounded}{R-tree splitting} 
%Quadratic method: Examine all the children of the overflowing node and find the pair of bounding boxes that would waste
% the most area were they to be inserted in the same node. This is determined by subtracting the sum of the areas of the
% two bounding boxes from the area of the covering bounding box. These two bounding boxes are placed in separate nodes,
% say j and k . The set of remaining bounding boxes are examined and the bounding box i whose addition maximizes the
% difference in coverage between the bounding boxes associated with j and k is added to the node whose coverage is
% minimized by the addition. This process is reapplied to the remaining bounding boxes [Gutt84]. This method takes
% quadratic time.
%\end{beamerboxesrounded}



\begin{frame}[fragile]
\begin{beamerboxesrounded}{Създаваме таблицата $map\_test$, където $loc$ е пространствен атрибут}
\begin{alltt}
mysql> create table map_test
    -> (
    ->   name varchar(100) not null primary key,
    ->   \alert{loc  geometry not null},
    -> );
Query OK, 0 rows affected (0.00 sec)
\end{alltt}
\end{beamerboxesrounded}
\end{frame}
\begin{frame}[fragile]
\begin{beamerboxesrounded}{Чрез следната процедурата добавяме 30 000 реда}
\tiny{\begin{alltt}
 mysql> CREATE PROCEDURE fill_points(IN size INT(10))
    -> BEGIN
    ->   DECLARE i DOUBLE(10,1) DEFAULT size;
    -> 
    ->   DECLARE lon FLOAT(7,4);
    ->   DECLARE lat FLOAT(6,4);
    ->   DECLARE position VARCHAR(100);
    -> 
    ->   DELETE FROM map_test;
    -> 
    ->   WHILE i > 0 DO
    ->     SET lon = RAND() * 360 - 180;
    ->     SET lat = RAND() * 180 - 90;
    -> 
    ->     SET position = CONCAT( 'POINT(', lon, ' ', lat, ')' );
    -> 
    ->     INSERT INTO map_test(name, loc) VALUES ( CONCAT('name_', i), GeomFromText(position) );
    -> 
    ->     SET i = i - 1;
    -> END WHILE;
    -> END @
Query OK, 0 rows affected (0.08 sec)

mysql> delimiter ;
mysql> CALL fill_points(30000);
Query OK, 1 row affected (26.00 sec)
\end{alltt}
}
\end{beamerboxesrounded}
\end{frame}

\begin{frame}[fragile]
\begin{beamerboxesrounded}{Заявка за проверка кои точки се съдържат в полигона}
\input{no_index.tex}
\end{beamerboxesrounded}
\end{frame}

\begin{frame}[fragile]
\begin{beamerboxesrounded}{Сега създаваме пространствен индекс по атрибута $loc$}
\begin{alltt}

mysql> create spatial index ps_index on map_test(loc);
Query OK, 30000 rows affected (2.44 sec)
Records: 30000  Duplicates: 0  Warnings: 0
\end{alltt}
\end{beamerboxesrounded}
\end{frame}

\begin{frame}[fragile]
\begin{beamerboxesrounded}{И отново правим същата заявка}
\input{index.tex}
\end{beamerboxesrounded}
\end{frame}

\end{document}
