\documentclass[a4paper,14pt]{article}
\usepackage{extsizes}
\usepackage[utf8]{inputenc}
\usepackage[T1]{fontenc}
\usepackage[bulgarian]{babel}
\usepackage{amsmath}
\usepackage [full]{textcomp}
\usepackage{graphicx}
\usepackage{fullpage}
\DeclareGraphicsExtensions{.pdf,.png,.jpg}


\title{
\hline
\vspace{0.5cm}
Курсов проект по
\\ \vspace{0.5cm} Системи за електронен бизнес
\vspace{0.5cm}
\hline
\\ \vspace{2cm}\Large{Тема: Галерия}}
\author{Валентина Динкова, ф.н. 71112, 2 група}

\begin{document}
\maketitle

\newpage

\section{Използвани среди и технологии}
За изготвянето на проекта са използвани ASP.NET и C\#. Методите, използвани за изготвянето на проекта са по образец от примера,
 показан в книгата книгата \textit{“Beginning ASP.NET E-Commerce in C\# 2005: From Novice to Professional”} на издателство \textit{Apress}.
 
 За целта съм използвала следния софтуер:
 \begin{itemize}
\item
Microsoft Visual Studio 2010
\item
SQL Server 2008
\item
SQL Server Management Studio Express
\end{itemize}

Използване е трислойна архитектура:
\begin{enumerate}
\item
\textbf{Презентационен слой (presentation tier)} – предоставя потребителския интерфейс; реализиран е на ASP.NET
\item
\textbf{Бизнес слой (business tier)} – слой посредник между другите два, реализира логиката на приложението. Той се програмира на езика C\#
\item
\textbf{Слой за съхранение и обработка на базата данни (data tier)} -  С помощтта на Stored Procedures бизнес слоя си комуникира с базата от данни.
\end{enumerate}

\section{Етапи на разработка}
Проектът за електронна галерия е изграден на няколко етапа:
\begin{enumerate}
\item
\textbf{Първи етап} - реализирани са категориите и подкатегориите на стоките, описанията на стоките, търсенето в каталога, отделите.
\item
\textbf{Втори етап} - реализирани са потребителската кошница, която съхранява данните за продуктите, добавени в нея в базата данни за всеки клиент, като има възможност за редактиране на съдържанието и и обработка на поръчките. Освен това се реализира и система за препоръки на стоки от магазина.

\item
\textbf{Трети етап} - реализира се частта за потребителските сметки, обработване на потребителските поръчки, проверка за наличност в склад.


\end{enumerate}



\end{document}